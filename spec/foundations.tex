
\documentclass[10pt]{article}
\usepackage[utf8]{inputenc}
\usepackage{amsmath,amssymb,amsthm}
\usepackage{mathrsfs}
\usepackage{natbib}
\usepackage{listings}
\usepackage{color}




\title{Logical Foundations}
\author{Michał Stanisław Wójcik}
\date{\today}


\theoremstyle{plain}
\newtheorem{theorem}{Theorem}[section]
\newtheorem{proposition}[theorem]{Proposition}
\newtheorem{definition}[theorem]{Definition}
\newtheorem{corollary}[theorem]{Corollary}
\newtheorem{lemma}[theorem]{Lemma}
\newtheorem{fact}[theorem]{Fact}
\newtheorem{conjecture}[theorem]{Conjecture}
\newtheorem{question}[theorem]{Question}
\newtheorem{claim}[theorem]{Claim}
\newtheorem{example}[theorem]{Example}
\newtheorem{assumption}[theorem]{Assumption}
\newtheorem{remark}[theorem]{Remark}
\newtheorem{axiom}[theorem]{Axiom}
\newtheorem{note}[theorem]{Note}
\newtheorem{law}{Law}


\newcommand{\defeq}{\stackrel{\mathrm{def}}{=}}
\newcommand{\elev}{\text{elev}}
\newcommand{\free}{\text{free}}
\newcommand{\bound}{\text{bound}}
\renewcommand{\axiom}{\text{Axiom}}
\renewcommand{\rule}{\text{Rule}}
\newcommand{\asm}{\text{Asm}}
\newcommand{\block}{\text{Block}}
\newcommand{\Sk}{\mathcal{S}}
\newcommand{\sk}{\mathfrak{s}}
\renewcommand{\i}{\mathfrak{i}}
\newcommand{\ii}{\hat{\mathfrak{i}}}
\newcommand{\N}{\mathbb{N}}
\newcommand{\F}{\mathcal{F}}
\newcommand{\A}{\mathcal{A}}
\newcommand{\B}{\mathcal{B}}
\newcommand{\C}{\mathcal{C}}
\renewcommand{\root}{\text{root}}
\newcommand{\final}{\text{final}}
\newcommand{\truth}{\mathfrak{t}}
\newcommand{\false}{\mathfrak{f}}
\newcommand{\context}{\text{Ctx}}
\newcommand{\Cnr}{\text{Cn}_\text{R}}
\renewcommand{\L}{\mathscr{L}}
\newcommand{\V}{\mathscr{V}}
\newcommand{\prove}{\vdash}
\newcommand{\M}{\mathbf{M}}
\renewcommand{\succ}{\text{S}}
\newcommand{\shift}{\text{shift}}
\newcommand{\fixed}{\text{fixed}}
\newcommand{\refset}{\text{ref}}
\newcommand{\arity}{\mathfrak{a}}


\definecolor{dkgreen}{rgb}{0,0.6,0}
\definecolor{gray}{rgb}{0.5,0.5,0.5}
\definecolor{mauve}{rgb}{0.58,0,0.82}


\lstset{frame=tb,
  language=Python,
  aboveskip=3mm,
  belowskip=3mm,
  showstringspaces=false,
  columns=flexible,
  basicstyle={\small\ttfamily},
  numbers=none,
  numberstyle=\tiny\color{gray},
  keywordstyle=\color{blue},
  commentstyle=\color{dkgreen},
  stringstyle=\color{mauve},
  breaklines=true,
  breakatwhitespace=true,
  tabsize=3
}
\setlength{\parindent}{0pt}


\begin{document}
\maketitle
\section{Well Formed Formula}

\subsection{Syntax}

Let $\V$ be an infinite countable set of variable symbols. And $\false$ be a special predicate symbol.

\begin{definition}
Let $\mathscr{P}$ be a set of predicate symbols for which $\false\in \mathscr{P}$ and $\mathscr{F}$ be a set of function symbols. Let $\arity : \mathscr{P} \cup \mathscr{F} \to \N$ be an arity function. 
then $\L \defeq (\mathscr{P}, \mathscr{F}, \arity)$ is a signature of a language.
\end{definition}

\begin{definition}[\textbf{arity}]
Let $\L \defeq (\mathscr{P}, \mathscr{F}, \arity)$ be a signature of a language.

\begin{equation}
\mathscr{P}_k \defeq \{p\in \mathscr{P}: \arity(p) = k \}, 
\end{equation}

\begin{equation}
\mathscr{F}_k \defeq \{f\in \mathscr{F}: \arity(f) = k \}. 
\end{equation}



\end{definition}

\begin{definition}[\textbf{recursive: term}]
Let $\L = (\mathscr{P}, \mathscr{F}, \arity)$ be a signature of a language.
\begin{enumerate}
\item For any $v\in \V$, $v$ is a \textbf{term}.
\item For any $c\in \mathscr{F}_0$, $c$ is a \textbf{term}.
\item For any $f\in \mathscr{F}_k$ and $\tau_1, \dots, \tau_k$ are terms, $f(\tau_1, \dots, \tau_k)$ is a \textbf{term}.
\end{enumerate}
\end{definition}

\begin{definition}[\textbf{recursive: atomic formula}]
Let $\L = (\mathscr{P}, \mathscr{F}, \arity)$ be a signature of a language.
\begin{enumerate}
\item For any $p\in \mathscr{P}_0$, $p$ is an \textbf{atomic formula}.
\item For any $p\in \mathscr{P}_k$ and $\tau_1, \dots, \tau_k$ are terms, $p(\tau_1, \dots, \tau_k)$ is an \textbf{atomic formula}.
\end{enumerate}

We will denote a set of all atomic formulas as $\A(\L)$.
\end{definition}


\begin{definition}[\textbf{recursive: base formula, well formed formula}]
Let $\L = (\mathscr{P}, \mathscr{F}, \arity)$ be a signature of a language.
\begin{enumerate}
\item For any atomic formula $\phi$, $\phi$ is a \textbf{base formula}.
\item For any \textbf{base formula} $\phi$, $\phi$ is a \textbf{well formed formula}.
\item For any \textbf{well formed formula} $\phi$, $(\phi)$ is a \textbf{base formula}.
\item For any \textbf{base formulas} $\alpha, \beta$, $\alpha \to \beta$ is a \textbf{well formed formula}.
\item For any $v\in \V$ and a \textbf{base formula} $\phi$, $\forall v. \phi$ is a \textbf{well formed formula}.
\end{enumerate}

We will denote set of all base formulas as $\B(\L)$ and set of all well formed formulas as $\F(\L)$.
\end{definition}

\begin{definition}[\textbf{recursive: variable in a term}]
Let $\L = (\mathscr{P}, \mathscr{F}, \arity)$ be a signature of a language.
Let $v\in \V$ and $\tau$ be a term.
\begin{enumerate}
\item If $\tau$ is $v$ then $v$ is a \textbf{variable in} term $\tau$.
\item If $\tau$ is $f(\tau_1, \dots \tau_k)$ where $f\in \mathscr{F}_k$ and $\tau_1, \dots \tau_k$ are terms and $v$ is a \textbf{variable in} some term $\tau_i$ where $i$ is an integer with $1 \leq i \leq k$, then $v$ is a \textbf{variable in} term $\tau$.
\item Otherwise $v$ is not a \textbf{variable in} a term $\tau$.
\end{enumerate}

\end{definition}

\begin{definition}[\textbf{recursive: variable in a well formed formula}]
Let $\L = (\mathscr{P}, \mathscr{F}, \arity)$ be a signature of a language.
Let $v\in \V$ and $\phi\in\F(\L)$.
\begin{enumerate}
\item If $\phi$ is $p(\tau_1, \dots, \tau_k)$ where $p\in \mathscr{P}_k$ and $\tau_1, \dots, \tau_k$ are terms, and $v$ is a \textbf{variable in} some term $\tau_i$ where $i$ is an integer with $1 \leq i \leq k$, then $v$ is a \textbf{variable in} $\phi$.
\item If $\phi$ is $\alpha \to \beta$ where $\alpha, \beta\in\B(\L)$, and $v$ is a \textbf{variable in} $\alpha$ or $v$ is a \textbf{variable in} $\beta$, then $v$ is a \textbf{variable in} $\phi$.
\item If $\phi$ is $\forall x. \psi$ where $x\in \V$ and $\psi\in\B(\L)$, and either $v$ is $x$ or $v$ is a \textbf{variable in} $\psi$, then $v$ is a \textbf{variable in} $\phi$.
\item Otherwise $v$ is not a \textbf{variable in} $\phi$.
\end{enumerate}

\end{definition}

\begin{definition}[\textbf{recursive: free variable in a well formed formula}]
Let $\L = (\mathscr{P}, \mathscr{F}, \arity)$ be a signature of a language.
Let $v\in \V$ and $\phi\in\F(\L)$.
\begin{enumerate}
\item If $\phi\in\A(\L)$ and $v$ is a variable in $\phi$, then $v$ is a \textbf{free variable in} $\phi$.
\item If $\phi$ is $\alpha \to \beta$ where $\alpha, \beta\in\B(\L)$, and $v$ is a \textbf{free variable in} $\alpha$ or $v$ is a \textbf{free variable in} $\beta$, then $v$ is a \textbf{free variable in} $\phi$.
\item If $\phi$ is $\forall x. \psi$ where $x\in \V$ and $\psi\in\B(\L)$ and $v$ is not $x$ and $v$ is a \textbf{free variable in} $\psi$, then $v$ is a \textbf{free variable in} $\phi$.
\item Otherwise $v$ is not a \textbf{free variable in} $\phi$.
\end{enumerate}

\end{definition}

\begin{definition}[\textbf{recursive: substitution in term}]
Let $\L = (\mathscr{P}, \mathscr{F}, \arity)$ be a signature of a language.
Let $v, x\in \V$ and $\tau, \sigma$ be terms.
\begin{enumerate}
\item If $\sigma$ is $x$, then $\sigma(x/\tau) \defeq \tau$.
\item If $\sigma$ is $v$ and $v$ is not $x$, then $\sigma(x/\tau) \defeq v$.
\item If $\sigma$ is $c$ where $c\in \mathscr{F}_0$, then $\sigma(x/\tau) \defeq c$.
\item If $\sigma$ is $f(\sigma_1, \dots, \sigma_k)$ where $f\in \mathscr{F}_k$ and $\sigma_1, \dots, \sigma_k$ are terms, then $\sigma(x/\tau) \defeq f(\sigma_1(x/\tau), \dots, \sigma_k(x/\tau))$.
\end{enumerate}

\end{definition}

\begin{definition}[\textbf{recursive: substitution in well formed formula}]
Let $\L = (\mathscr{P}, \mathscr{F}, \arity)$ be a signature of a language.
Let $x\in \V$, $\tau$ be a term, and $\phi\in\F(\L)$.
\begin{enumerate}
\item If $\phi$ is $p$ where $p\in \mathscr{P}_0$, then $\phi(x/\tau) \defeq p$.
\item If $\phi$ is $p(\sigma_1, \dots, \sigma_k)$ where $p\in \mathscr{P}_k$ and $\sigma_1, \dots, \sigma_k$ are terms, then $\phi(x/\tau) \defeq p(\sigma_1(x/\tau), \dots, \sigma_k(x/\tau))$.
\item If $\phi$ is $\alpha \to \beta$ where $\alpha, \beta\in\B(\L)$, then $\phi(x/\tau) \defeq \alpha(x/\tau) \to \beta(x/\tau)$.
\item If $\phi$ is $\forall y. \psi$ where $y\in \V$ and $\psi\in\B(\L)$ and $y$ is $x$, then $\phi(x/\tau) \defeq \forall y. \psi$.
\item If $\phi$ is $\forall y. \psi$ where $y\in \V$ and $\psi\in\B(\L)$ and $y$ is not $x$ and $y$ is not a variable in $\tau$, then $\phi(x/\tau) \defeq \forall y. \psi(x/\tau)$.
\item Otherwise $\phi(x/\tau)$ is \textbf{not admissible}.
\end{enumerate}

\end{definition}

\section{Semantics}

\begin{definition} [\textbf{$\L$-structure}]
Let $\L = (\mathscr{P}, \mathscr{F}, \arity)$ be a signature of a language and $A$ be an arbitrary mathematical domain, then a mapping $\M$ such that 
\begin{equation}
\M: \mathscr{P} \cup \mathscr{F} \to \{0, 1\} \cup \bigcup_{k=1}^\infty \{0, 1\}^{(A^k)} \cup A
\cup \bigcup_{k=1}^\infty A^{(A^k)},
\end{equation}
where $\M(\false) = 0$,
\begin{equation}
\M(\mathscr{P}_0) \subset \{0, 1\} \text{ and } \M(\mathscr{P}_k) \subset \{0, 1\}^{(A^k)} \text{ for } k = 1, \dots,
\end{equation}
and
\begin{equation}
\M(\mathscr{F}_0)\subset A \text{ and } \M(\mathscr{F}_k) \subset A^{(A^k)}
\text{ for } k = 1, \dots
\end{equation}
is called an $\L$-structure with domain $A$.\\

For convenience we will denote values of $\M$ as $p^\M \defeq \M(p)$ for $p\in\mathscr{P}$ and $f^\M \defeq \M(f)$ for $f\in\mathscr{F}$.
\end{definition}

\begin{definition}[\textbf{assignment}]
Let $A$ be an arbitrary mathematical domain. Then $j: \V \to A$ is called an variables assignment in $A$.
\end{definition}

\begin{definition}[\textbf{recursive: term evaluation}]
Let $\L = (\mathscr{P}, \mathscr{F}, \arity)$ be a signature of a language and $A$ be an arbitrary mathematical domain, let $\M$ be an $\L$-structure and $j:\V\to A$. Let $\tau$ be an arbitrary term.
\begin{enumerate}
\item If $\tau$ is $v$ where $v\in\V$, then $\tau^{\M,j} \defeq j(v)$.
\item If $\tau$ is $f$ where $f\in\F_0$, then $\tau^{\M,j} \defeq f^\M$.
\item If $\tau$ is $f(\tau_1, \dots, \tau_k)$ where $f\in\F_k$ and $\tau_1, \dots, \tau_k$ 
are terms, then $\tau^{\M,j} \defeq f^\M(\tau_1^{\M, j}, \dots, \tau_k^{\M,j})$.
\end{enumerate}
\end{definition}

\begin{definition}
Let $\L$ be a signature of a language.
Let $\M$ be a $\L$-structure with domain $A$ and let $j:\V\to A$ be an assignment of variables. Then we define  


\begin{equation}
j\cfrac{a}{x}(v) \defeq
\begin{cases}
a \text{ if } v \text{ is } x,\\
j(v) \text{ otherwise}.
\end{cases}
\end{equation}
\end{definition}

For the purpouses of this work, we will define product of infinite number of 0s and 1s.
\begin{definition}
Let $S$ be an arbitrary set and $a_s\in\{0, 1\}$ for any $s\in S$.
\begin{equation}
\prod_{s\in S} a_s = 
\begin{cases}
1 \text{ if } a_s = 1 \text{ for all } s\in S, \\
0 \text{ otherwise }.
\end{cases}
\end{equation}
\end{definition}

\begin{definition} [\textbf{recursive: formula evaluation}]
Let $\L = (\mathscr{P}, \mathscr{F}, \arity)$ be a signature of a language and $A$ be an arbitrary mathematical domain, let $\M$ be an $\L$-structure with domain $A$ and $j:\V\to A$. Let $\phi\in\F(\L)$.
\begin{enumerate}
\item If $\phi$ is $p$ where $p\in \mathscr{P}$, then $\phi^{\M, j} \defeq p^\M$.
\item If $\phi$ is $p(\tau_1, \dots, \tau_k)$ where $p\in\mathscr{P}_k$ and $\tau_1, \dots, \tau_k$ are terms, then $\phi^{\M, j} \defeq p^\M(\tau_1^{\M, j}, \dots, \tau_k^{\M,j})$.
\item If $\phi$ is $\alpha \to \beta$ where $\alpha, \beta\in\B(\L)$, then
\begin{equation}
\phi^{\M, j} \defeq 1 - \alpha^{\M, j}(1 - \beta^{\M, j}).
\end{equation}
\item If $\phi$ is $\forall x. \psi$ where $x\in \V$ and $\psi\in\B(\L)$, then
\begin{equation}
\phi^{\M, j} \defeq \prod_{a\in A} \psi^{\M, j\cfrac{a}{x}}.
\end{equation}
 
\end{enumerate}

\end{definition}

\begin{corollary}
Let $\L = (\mathscr{P}, \mathscr{F}, \arity)$ be a signature of a language and $A$ be an arbitrary mathematical domain, let $\M$ be an $\L$-structure with domain $A$ and $j:\V\to A$. Let $\phi\in\F(\L)$, then
\begin{equation}
(\phi \to \false)^{\M, j} = 1 - \phi^{\M, j}.
\end{equation}
\end{corollary}

\begin{definition} [\textbf{model}]
Let $\L = (\mathscr{P}, \mathscr{F}, \arity)$ be a signature of a language and $A$ be an arbitrary mathematical domain, let $\M$ be an $\L$-structure with domain $A$ and $j:\V\to A$. Let $\phi\in\F(\L)$, then

\begin{equation}
\M, j\models \phi \text{ iff } \phi^{\M, j} = 1.
\end{equation}

\begin{theorem}
\label{admissible-theorem}
Let $\L$ be a signature of a language. Let $\phi\in\F(\L)$ and let $\tau$ be an arbitrary term. If $\phi(x/\tau)$ is admissible, then for any $\L$-structure $\M$ with domain $A$ and for any assignment of variables $j:\V\to A$ we have
\begin{equation}
(\M, j\cfrac{\tau^{(\M, j)}}{x})\models \phi \text{ iff } (\M, j)\models \phi(x/\tau).
\end{equation}
\end{theorem}

\begin{proof}
We proceed by structural induction on $\phi\in\F(\L)$.

\textbf{Case 1:} $\phi = p$ where $p\in\mathscr{P}_0$.

By the definition of substitution in well formed formula, $\phi(x/\tau) = p$. By the definition of formula evaluation, $\phi^{\M, j} = p^\M$ for any assignment $j$. Thus
\begin{align*}
(\M, j\cfrac{\tau^{(\M, j)}}{x})\models \phi &\iff \phi^{\M, j\cfrac{\tau^{(\M, j)}}{x}} = 1 \\
&\iff p^\M = 1 \\
&\iff \phi^{\M, j} = 1 \\
&\iff (\M, j)\models \phi \\
&\iff (\M, j)\models \phi(x/\tau).
\end{align*}

\textbf{Case 2:} $\phi = p(\sigma_1, \dots, \sigma_k)$ where $p\in\mathscr{P}_k$ and $\sigma_1, \dots, \sigma_k$ are terms.

By the definition of substitution in well formed formula, $\phi(x/\tau) = p(\sigma_1(x/\tau), \dots, \sigma_k(x/\tau))$. We need to show that for each term $\sigma_i$, we have $\sigma_i^{\M, j\cfrac{\tau^{(\M, j)}}{x}} = (\sigma_i(x/\tau))^{\M, j}$.

We prove this by structural induction on terms:
\begin{itemize}
\item If $\sigma_i = x$, then $\sigma_i(x/\tau) = \tau$ and $\sigma_i^{\M, j\cfrac{\tau^{(\M, j)}}{x}} = (j\cfrac{\tau^{(\M, j)}}{x})(x) = \tau^{\M, j} = (\sigma_i(x/\tau))^{\M, j}$.
\item If $\sigma_i = v$ where $v\neq x$, then $\sigma_i(x/\tau) = v$ and $\sigma_i^{\M, j\cfrac{\tau^{(\M, j)}}{x}} = (j\cfrac{\tau^{(\M, j)}}{x})(v) = j(v) = v^{\M, j} = (\sigma_i(x/\tau))^{\M, j}$.
\item If $\sigma_i = c$ where $c\in\mathscr{F}_0$, then $\sigma_i(x/\tau) = c$ and $\sigma_i^{\M, j\cfrac{\tau^{(\M, j)}}{x}} = c^\M = (\sigma_i(x/\tau))^{\M, j}$.
\item If $\sigma_i = f(\sigma_{i,1}, \dots, \sigma_{i,m})$ where $f\in\mathscr{F}_m$, then by inductive hypothesis on terms, $\sigma_{i,\ell}^{\M, j\cfrac{\tau^{(\M, j)}}{x}} = (\sigma_{i,\ell}(x/\tau))^{\M, j}$ for all $\ell$. Thus
\begin{align*}
\sigma_i^{\M, j\cfrac{\tau^{(\M, j)}}{x}} &= f^\M(\sigma_{i,1}^{\M, j\cfrac{\tau^{(\M, j)}}{x}}, \dots, \sigma_{i,m}^{\M, j\cfrac{\tau^{(\M, j)}}{x}}) \\
&= f^\M((\sigma_{i,1}(x/\tau))^{\M, j}, \dots, (\sigma_{i,m}(x/\tau))^{\M, j}) \\
&= (f(\sigma_{i,1}(x/\tau), \dots, \sigma_{i,m}(x/\tau)))^{\M, j} \\
&= (\sigma_i(x/\tau))^{\M, j}.
\end{align*}
\end{itemize}

Therefore,
\begin{align*}
(\M, j\cfrac{\tau^{(\M, j)}}{x})\models \phi &\iff \phi^{\M, j\cfrac{\tau^{(\M, j)}}{x}} = 1 \\
&\iff p^\M(\sigma_1^{\M, j\cfrac{\tau^{(\M, j)}}{x}}, \dots, \sigma_k^{\M, j\cfrac{\tau^{(\M, j)}}{x}}) = 1 \\
&\iff p^\M((\sigma_1(x/\tau))^{\M, j}, \dots, (\sigma_k(x/\tau))^{\M, j}) = 1 \\
&\iff (p(\sigma_1(x/\tau), \dots, \sigma_k(x/\tau)))^{\M, j} = 1 \\
&\iff (\phi(x/\tau))^{\M, j} = 1 \\
&\iff (\M, j)\models \phi(x/\tau).
\end{align*}

\textbf{Case 3:} $\phi = \alpha \to \beta$ where $\alpha, \beta\in\B(\L)$.

By the definition of substitution in well formed formula, $\phi(x/\tau) = \alpha(x/\tau) \to \beta(x/\tau)$. By inductive hypothesis,
\begin{align*}
(\M, j\cfrac{\tau^{(\M, j)}}{x})\models \alpha &\iff (\M, j)\models \alpha(x/\tau), \\
(\M, j\cfrac{\tau^{(\M, j)}}{x})\models \beta &\iff (\M, j)\models \beta(x/\tau).
\end{align*}

By the definition of formula evaluation,
\begin{align*}
(\M, j\cfrac{\tau^{(\M, j)}}{x})\models \phi &\iff \phi^{\M, j\cfrac{\tau^{(\M, j)}}{x}} = 1 \\
&\iff 1 - \alpha^{\M, j\cfrac{\tau^{(\M, j)}}{x}}(1 - \beta^{\M, j\cfrac{\tau^{(\M, j)}}{x}}) = 1 \\
&\iff 1 - (\alpha(x/\tau))^{\M, j}(1 - (\beta(x/\tau))^{\M, j}) = 1 \\
&\iff (\alpha(x/\tau) \to \beta(x/\tau))^{\M, j} = 1 \\
&\iff (\phi(x/\tau))^{\M, j} = 1 \\
&\iff (\M, j)\models \phi(x/\tau).
\end{align*}

\textbf{Case 4:} $\phi = \forall y. \psi$ where $y\in\V$ and $\psi\in\B(\L)$.

\textbf{Subcase 4a:} If $y = x$, then by the definition of substitution, $\phi(x/\tau) = \forall y. \psi$. Thus
\begin{align*}
(\M, j\cfrac{\tau^{(\M, j)}}{x})\models \phi &\iff \phi^{\M, j\cfrac{\tau^{(\M, j)}}{x}} = 1 \\
&\iff \prod_{a\in A} \psi^{\M, (j\cfrac{\tau^{(\M, j)}}{x})\cfrac{a}{y}} = 1 \\
&\iff \prod_{a\in A} \psi^{\M, j\cfrac{a}{y}} = 1 \quad \text{(since $y = x$ and $(j\cfrac{\tau^{(\M, j)}}{x})\cfrac{a}{x} = j\cfrac{a}{x}$)} \\
&\iff \phi^{\M, j} = 1 \\
&\iff (\M, j)\models \phi \\
&\iff (\M, j)\models \phi(x/\tau).
\end{align*}

\textbf{Subcase 4b:} If $y\neq x$ and $y$ is not a \textbf{variable in} $\tau$, then by the definition of substitution, $\phi(x/\tau) = \forall y. \psi(x/\tau)$. By inductive hypothesis, for any $a\in A$,
$$(\M, j\cfrac{a}{y}\cfrac{\tau^{(\M, j\cfrac{a}{y})}}{x})\models \psi \iff (\M, j\cfrac{a}{y})\models \psi(x/\tau).$$

Since $y$ is not a \textbf{variable in} $\tau$, the evaluation of $\tau$ does not depend on the value assigned to $y$. Therefore, $\tau^{(\M, j\cfrac{a}{y})} = \tau^{(\M, j)}$ for all $a\in A$.

Moreover, since $x\neq y$, we have $j\cfrac{\tau^{(\M, j)}}{x}\cfrac{a}{y} = j\cfrac{a}{y}\cfrac{\tau^{(\M, j)}}{x}$ for all $a\in A$.

Thus,
\begin{align*}
(\M, j\cfrac{\tau^{(\M, j)}}{x})\models \phi &\iff \phi^{\M, j\cfrac{\tau^{(\M, j)}}{x}} = 1 \\
&\iff \prod_{a\in A} \psi^{\M, (j\cfrac{\tau^{(\M, j)}}{x})\cfrac{a}{y}} = 1 \\
&\iff \prod_{a\in A} \psi^{\M, j\cfrac{a}{y}\cfrac{\tau^{(\M, j)}}{x}} = 1 \\
&\iff \prod_{a\in A} (\psi(x/\tau))^{\M, j\cfrac{a}{y}} = 1 \quad \text{(by inductive hypothesis)} \\
&\iff (\forall y. \psi(x/\tau))^{\M, j} = 1 \\
&\iff (\phi(x/\tau))^{\M, j} = 1 \\
&\iff (\M, j)\models \phi(x/\tau).
\end{align*}

This completes the proof by structural induction.
\end{proof}

\end{definition}

\end{document}