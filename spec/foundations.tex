
\documentclass[10pt]{book}
\usepackage[a4paper, margin=1.5in, centering]{geometry}
\usepackage[utf8]{inputenc}
\usepackage{amsmath,amssymb,amsthm}
\usepackage{bm}
\usepackage{mathrsfs}
\usepackage{natbib}
\usepackage{listings}
\usepackage{color}
\usepackage{hyperref}




\title{Logical Foundations}
\author{Michał Stanisław Wójcik}
\date{\today}


\theoremstyle{plain}
\newtheorem{theorem}{Theorem}[section]
\newtheorem{proposition}[theorem]{Proposition}
\newtheorem{definition}[theorem]{Definition}
\newtheorem{corollary}[theorem]{Corollary}
\newtheorem{lemma}[theorem]{Lemma}
\newtheorem{fact}[theorem]{Fact}
\newtheorem{conjecture}[theorem]{Conjecture}
\newtheorem{question}[theorem]{Question}
\newtheorem{claim}[theorem]{Claim}
\newtheorem{example}[theorem]{Example}
\newtheorem{assumption}[theorem]{Assumption}
\newtheorem{remark}[theorem]{Remark}
\newtheorem{axiom}[theorem]{Axiom}
\newtheorem{note}[theorem]{Note}
\newtheorem{law}{Law}


\newcommand{\defeq}{\stackrel{\mathrm{def}}{=}}
\newcommand{\defiff}{\stackrel{\mathrm{def}}{\iff}}
\newcommand{\dis}[2]{{#1}^{\nabla #2}}
\newcommand{\sxeq}{\doteq}
\newcommand{\elev}{\text{elev}}
\newcommand{\free}{\text{free}}
\newcommand{\liff}{\leftrightarrow}
\newcommand{\var}{\text{var}}
\newcommand{\bound}{\text{bound}}
\renewcommand{\axiom}{\text{Axiom}}
\renewcommand{\rule}{\text{Rule}}
\newcommand{\asm}{\text{Asm}}
\newcommand{\block}{\text{Block}}
\newcommand{\Sk}{\mathcal{S}}
\newcommand{\sk}{\mathfrak{s}}
\renewcommand{\i}{\mathfrak{i}}
\newcommand{\ii}{\hat{\mathfrak{i}}}
\newcommand{\N}{\mathbb{N}}
\newcommand{\F}{\mathcal{F}}
\newcommand{\A}{\mathcal{A}}
\newcommand{\B}{\mathcal{B}}
\newcommand{\C}{\mathcal{C}}
\renewcommand{\S}{\mathcal{S}}
\newcommand{\T}{\mathcal{T}}
\renewcommand{\root}{\text{root}}
\newcommand{\final}{\text{final}}
\newcommand{\truth}{\mathfrak{t}}
\newcommand{\false}{\mathfrak{f}}
\newcommand{\context}{\text{Ctx}}
\newcommand{\Cnr}{\text{Cn}_\text{R}}
\renewcommand{\L}{\mathscr{L}}
\newcommand{\V}{\mathscr{V}}
\newcommand{\prove}{\vdash}
\newcommand{\M}{\mathbf{M}}
\renewcommand{\succ}{\text{S}}
\newcommand{\shift}{\text{shift}}
\newcommand{\fixed}{\text{fixed}}
\newcommand{\refset}{\text{ref}}
\newcommand{\arity}{\mathfrak{a}}


\definecolor{dkgreen}{rgb}{0,0.6,0}
\definecolor{gray}{rgb}{0.5,0.5,0.5}
\definecolor{mauve}{rgb}{0.58,0,0.82}


\lstset{frame=tb,
  language=Python,
  aboveskip=3mm,
  belowskip=3mm,
  showstringspaces=false,
  columns=flexible,
  basicstyle={\small\ttfamily},
  numbers=none,
  numberstyle=\tiny\color{gray},
  keywordstyle=\color{blue},
  commentstyle=\color{dkgreen},
  stringstyle=\color{mauve},
  breaklines=true,
  breakatwhitespace=true,
  tabsize=3
}
\setlength{\parindent}{0pt}


\begin{document}
\maketitle



\chapter{Introduction}

\section{Acknowledgements}

This document presents the logical foundations for a formal system based on first-order logic. The net of definitions and theorems presented here were designed by a human author - Michał Stanisław Wójcik. Claude AI assisted with writing proofs and definitions following the author's specifications and design. All AI-generated content in this paper has been carefully reviewed, verified and corrected by Michał Stanisław Wójcik who takes the full responsibility for the content of this work.

\tableofcontents
\newpage

\chapter{First Order Logic}

\section{Preliminaries}

We will use $\N$ to denote a set of all nonnegative integers.

\begin{definition}
\label{set-of-sequences}
Let $S$ be an arbitrary set.
\begin{enumerate}
\item $S^0 \defeq \{\emptyset\}$.
\item $S^k \defeq \underbrace{S \times \cdots \times S}_{k}$. 
\item $S^* \defeq \bigcup_{k=0}^\infty S^k$.
\end{enumerate}
\end{definition}

Note that $\emptyset$ is treated as empty sequence.

\begin{remark}
For a sequence $\bm{s}\in S^k$, we denote the $i$-th element of the sequence by $s_i$ where $i \in \{1, \dots, k\}$. let also $|\bm{s}| \defeq k$ denote a length of $\bm{s}$.
\end{remark}

Note that according to the above convention $|\emptyset| = 0$.



\begin{definition}
\label{dis-set-of-sequences}
Let $S$ be an arbitrary set.
\begin{enumerate}
\item $\dis{S}{0} \defeq \{\emptyset\}$.
\item $\dis{S}{k} 
\defeq \{\bm{s}\in S^k: \bm{s}_i \not=\bm{s}_j \text{ for } i \not= j\}$. 
\item $\dis{S}{} \defeq \bigcup_{k=0}^\infty \dis{S}{k}$.
\end{enumerate}

\end{definition}


\section{First Order Logic Formula}

\subsection{Syntax}



Let $\mathfrak{L}$ denote a set of all logical symbols `` $($ ", `` $)$ ", `` $\to$ ", `` $.$ ", ``$,$ ", `` $\forall$ ".
Let $\V$ be an infinite countable set of variable symbols. And $\false$ be a special predicate symbol.

\begin{definition}
Let $\mathscr{P}$ be a set of predicate symbols for which $\false\in \mathscr{P}$ and $\mathscr{F}$ be a set of function symbols. Let $\arity : \mathscr{P} \cup \mathscr{F} \to \N$ be an arity function. 
then $\L \defeq (\mathscr{P}, \mathscr{F}, \arity)$ is a signature of a language.
\end{definition}

To write first order logic formulas for a given language signature $\L$, we will use symbols from
\begin{equation}
\S(\L) \defeq \mathfrak{L} \cup \V \cup \mathscr{P} \cup \mathscr{F}.
\end{equation}

All formulas will be then a subset of $\S(\L)^*$.
Elements of $\S(\L)^*$ will be called strings.

\begin{note}[\textbf{syntactic equality}]
We use the notation $\sxeq$ to denote equality of strings. We will sometimes overload it when cast of some object to string is obvious.
\end{note}



When we stack symbols together from left to right, we mean that we concatenate them in the same order. E.g let $p$ be a predicate symbol and $x, y$ be variable symbols, then $p(x,y)$ is just a sequence of 6 symbols: `` $p$ ", ``$($", `` $x$ ", `` $,$ ", `` $y$ ", `` $)$ ".\\

For convenience we will use an abreviated style of writing, in which for any symbol 
$s\in \mathscr{P} \cup \mathscr{F}$ and for any seqence of strings $\bm{x}\in \S(\L)^*$

\begin{equation}
s(\bm{x}) \sxeq
\begin{cases}
s(x_1, \dots, x_k) \text{ for } |\bm{x}| > 0,\\
s \text{ for } |\bm{x}| = 0.
\end{cases}
\end{equation}

\begin{definition}[\textbf{arity}]
Let $\L \defeq (\mathscr{P}, \mathscr{F}, \arity)$ be a signature of a language.

\begin{equation}
\mathscr{P}_k \defeq \{p\in \mathscr{P}: \arity(p) = k \}, 
\end{equation}

\begin{equation}
\mathscr{F}_k \defeq \{f\in \mathscr{F}: \arity(f) = k \}. 
\end{equation}
\end{definition}

\begin{definition}[\textbf{atomic term}]
Let $\L = (\mathscr{P}, \mathscr{F}, \arity)$ be a signature of a language.
\begin{enumerate}
\item For any $v\in \V$, $v$ is an \textbf{atomic term}.
\item For any $c\in \mathscr{F}_0$, $c$ is an \textbf{atomic term}.
\end{enumerate}
\end{definition}

\begin{definition}[\textbf{recursive: term}]
Let $\L = (\mathscr{P}, \mathscr{F}, \arity)$ be a signature of a language.
We will define recursively a set of all terms $\T(\L)$.\\

Let $s\in \S(\L)^*$.
\begin{enumerate}
\item If $s \sxeq v$ where $v\in \V$, then $s\in \T(\L)$.
\item If $s \sxeq f(\bm{\tau})$ where $\bm{\tau}\in \T(\L)^*$ and $f\in \mathscr{F}_{|\bm{\tau}|}$, then $s\in \T(\L)$.
\item Otherwise $s\not\in \T(\L)$.
\end{enumerate}
\end{definition}

\begin{definition}[\textbf{atomic formula}]
Let $\L = (\mathscr{P}, \mathscr{F}, \arity)$ be a signature of a language.
We will define a set of all atomic formulas $\A(\L)$.

\begin{equation}
\A(\L) \defeq \bigcup_{k=0}^\infty \{p(\bm{\tau}): p\in\mathscr{P}_k, \bm{\tau}\in \T(\L)^k\}.
\end{equation}

\end{definition}

\begin{definition}[\textbf{recursive: base formula, well-formed formula}]
Let $\L = (\mathscr{P}, \mathscr{F}, \arity)$ be a signature of a language.
We will define recursively a set of all base formulas $\B(\L)$ and a set of all well-formed formulas $\F(\L)$.\\

Let $s\in \S(\L)^*$.

\begin{enumerate}
\item If $s\in \A(\L)$, then $s\in \B(\L)$.
\item If $s\in\B(\L)$, then $s\in \F(\L)$.
\item If $s \sxeq (\phi)$ where $\phi \in \F(\L)$, then $s \in \B(\L)$.
\item If $s \sxeq \alpha \to \beta$ where $\alpha, \beta\in \B(\L)$, then $s\in \F(\L)$.
\item If $s \sxeq \forall v. \phi$ where $\phi\in\B(\L)$, then $s\in \F(\L)$.
\item Otherwise $s\not\in\F(\L)$ and $s\not\in\B(\L)$.
\end{enumerate}
\end{definition}

\begin{definition}[\textbf{recursive: variable in term}]
Let $\L = (\mathscr{P}, \mathscr{F}, \arity)$ be a signature of a language.
We will define recursively a set $\var(\tau)$ of all variables in a term $\tau$.\\

Let $v\in \V$ and $\tau\in\T(\L)$.
\begin{enumerate}
\item If $\tau \sxeq v$ then $v\in\var(\tau)$.
\item If $\tau \sxeq f(\bm{\sigma})$ where $\bm{\sigma}\in\T(\L)^*$ and $f\in \mathscr{F}_{|\bm{\sigma}|}$ and 
$v\in\var(\bm{\sigma}) \defeq \bigcup_{k=1}^{|\bm{\sigma}|} \var(\sigma_k)$, then $v\in\var(\tau)$.
\item Otherwise $v\not\in\var(\tau)$.
\end{enumerate}

\end{definition}

\begin{definition}[\textbf{recursive: variable in formula}]
Let $\L = (\mathscr{P}, \mathscr{F}, \arity)$ be a signature of a language.
We will define recursively a set $\var(\phi)$ of all variables in a formula $\phi$.\\

Let $v\in \V$ and $\phi\in\F(\L)$.
\begin{enumerate}
\item If $\phi \sxeq p(\bm{\tau})$ where 
$\bm{\tau}\in\T(\L)^*$ and $p\in\mathscr{P}_{|\bm{\tau}|}$ and $v\in\var(\bm{\tau})$, then $v\in\var(\phi)$.
\item If $\phi \sxeq \alpha \to \beta$ where $\alpha, \beta\in\B(\L)$, and $v\in\var(\alpha)$ or $v\in\var(\beta)$, then $v\in\var(\phi)$.
\item If $\phi \sxeq \forall x. \psi$ where $x\in \V$ and $\psi\in\B(\L)$, and either $v = x$ or $v\in\var(\psi)$, then $v\in\var(\phi)$.
\item Otherwise $v\not\in\var(\phi)$.
\end{enumerate}

\end{definition}


\begin{definition}[\textbf{recursive: free variable in formula}]
Let $\L = (\mathscr{P}, \mathscr{F}, \arity)$ be a signature of a language.
We will define recursively a set $\free(\phi)$ of all free variables in a formula $\phi$.\\

Let $v\in \V$ and $\phi\in\F(\L)$.
\begin{enumerate}
\item If $\phi\in\A(\L)$ and $v\in\var(\phi)$, then $v\in\free(\phi)$.
\item If $\phi \sxeq \alpha \to \beta$ where $\alpha, \beta\in\B(\L)$, and $v\in\free(\alpha)$ or $v\in\free(\beta)$, then $v\in\free(\phi)$.
\item If $\phi \sxeq \forall x. \psi$ where $\psi\in\B(\L)$, $x\in \V$ and $x\not=v$, and $v\in\free(\psi)$, then $v\in\free(\phi)$.
\item Otherwise $v\not\in\free(\phi)$.
\end{enumerate}

\end{definition}


\subsection{Semantics}

\begin{definition} [\textbf{$\L$-structure}]
Let $\L = (\mathscr{P}, \mathscr{F}, \arity)$ be a signature of a language and $A\not=\emptyset$ be an arbitrary mathematical domain, then a mapping $\M$ such that 
\begin{equation}
\M: \mathscr{P} \cup \mathscr{F} \to \bigcup_{k=0}^\infty \{0, 1\}^{(A^k)} \cup \bigcup_{k=0}^\infty A^{(A^k)},
\end{equation}
where 
\begin{align*}
&\M(\false) = (\emptyset \mapsto 0),\\
&\M(\mathscr{P}_k) \subset \{0, 1\}^{(A^k)} \text{ for } k\in\N,\\
&\M(\mathscr{F}_k) \subset A^{(A^k)} \text{ for } k\in\N.\\
\end{align*}

is called an $\L$-structure with domain $A$.\\

For convenience we will denote values of $\M$ as $p^\M \defeq \M(p)$ for $p\in\mathscr{P}$ and $f^\M \defeq \M(f)$ for $f\in\mathscr{F}$.
Note that $p^\M:A^k\to\{0, 1\}$ and $f^\M:A^k\to A$.
\end{definition}

\begin{definition}[\textbf{assignment}]
Let $A$ be an arbitrary mathematical domain. Then $j: \V \to A$ is called a variables assignment in $A$.
\end{definition}

\begin{definition}[\textbf{recursive: term evaluation}]
\label{term-eval-def}
Let $\L = (\mathscr{P}, \mathscr{F}, \arity)$ be a signature of a language and $A$ be an arbitrary mathematical domain, let $\M$ be an $\L$-structure and $j:\V\to A$.\\ 

Let $\tau\in\T(\L)$.
\begin{enumerate}
\item If $\tau \sxeq v$ where $v\in\V$, then $\tau^{\M,j} \defeq j(v)$.
\item If $\tau \sxeq f(\bm{\sigma})$ where $\bm{\sigma}\in \T(\L)^*$
and $f\in\F_{|\bm{\sigma}|}$, then $\tau^{\M,j} \defeq f^\M(\bm{\sigma}^{\M, j})$ 
where 
\begin{equation}
\bm{\sigma}^{\M, j} \defeq 
\begin{cases}
(\sigma_1^{\M, j}, \dots, \sigma_k^{\M, j}) \text{ for } k = |\bm{\sigma}| > 0,\\
\emptyset \text{ for } |\bm{\sigma}| = 0.
\end{cases}
\end{equation}

\end{enumerate}
\end{definition}

\begin{definition}
Let $\L$ be a signature of a language.
Let $\M$ be a $\L$-structure with domain $A$ and let $j:\V\to A$ be an assignment of variables. Then we define  


\begin{equation}
j\cfrac{a}{x}(v) \defeq
\begin{cases}
a \text{ if } v = x,\\
j(v) \text{ otherwise}.
\end{cases}
\end{equation}
\end{definition}

\begin{definition}
Let $\L$ be a signature of a language.
Let $\M$ be a $\L$-structure with domain $A$ and let $j:\V\to A$ be an assignment of variables. 
Let $k$ be a positive integer and 
let $\bm{x}\in\dis{V}{k}$, and $\bm{a}\in A^k$.

Then we define  
\begin{equation}
j\cfrac{\bm{a}}{\bm{x}}(v) \defeq
\begin{cases}
a_i \text{ if } v = x_i \text{ for some } i\in\{1, \dots, k\},\\
j(v) \text{ otherwise}.
\end{cases}
\end{equation}
\end{definition}

For the purpouses of this work, we will define multiplication product of infinite number of 0s and/or 1s.
\begin{definition}
Let $S$ be an arbitrary set and $a_s\in\{0, 1\}$ for any $s\in S$.
\begin{equation}
\prod_{s\in S} a_s = 
\begin{cases}
1 \text{ if } a_s = 1 \text{ for all } s\in S, \\
0 \text{ otherwise }.
\end{cases}
\end{equation}
\end{definition}

\begin{definition} [\textbf{recursive: formula evaluation}]
Let $\L = (\mathscr{P}, \mathscr{F}, \arity)$ be a signature of a language and $A$ be an arbitrary mathematical domain, let $\M$ be an $\L$-structure with domain $A$ and $j:\V\to A$. Let $\phi\in\F(\L)$.
\begin{enumerate}
\item If $\phi \sxeq p(\bm{\tau})$ where 
$\bm{\tau}\in\T(\L)^*$ and $p\in\mathscr{P}_{|\bm{\tau}|}$, then $\phi^{\M, j} \defeq p^\M(\bm{\tau}^{\M, j})$.
\item If $\phi \sxeq \alpha \to \beta$ where $\alpha, \beta\in \B(\L)$, then
\begin{equation}
\phi^{\M, j} \defeq 1 - \alpha^{\M, j}(1 - \beta^{\M, j}).
\end{equation}
\item If $\phi \sxeq \forall x. \psi$ where $x\in \V$ and $\psi\in\B(\L)$, then
\begin{equation}
\phi^{\M, j} \defeq \prod_{a\in A} \psi^{\M, j\cfrac{a}{x}}.
\end{equation}

\end{enumerate}

\end{definition}

\begin{corollary}
Let $\L = (\mathscr{P}, \mathscr{F}, \arity)$ be a signature of a language and $A$ be an arbitrary mathematical domain, let $\M$ be an $\L$-structure with domain $A$ and $j:\V\to A$. Let $\phi\in\F(\L)$, then
\begin{equation}
(\phi \to \false)^{\M, j} = 1 - \phi^{\M, j}.
\end{equation}
\end{corollary}

\begin{definition} [\textbf{interpretation}]
Let $\L = (\mathscr{P}, \mathscr{F}, \arity)$ be a signature of a language and $A$ be an arbitrary mathematical domain, let $\M$ be an $\L$-structure with domain $A$ and $j:\V\to A$. Let $\phi\in\F(\L)$, then

\begin{equation}
\M, j\models \phi \defiff \phi^{\M, j} = 1.
\end{equation}

\end{definition}

\begin{definition} [\textbf{model}]
Let $\L = (\mathscr{P}, \mathscr{F}, \arity)$ and let $\M$ be an $\L$-structure. Let $\phi\in\F(\L)$.
We define

\begin{equation}
\M \models \phi
\end{equation}

iff for any assignment $j:\V\to A$, we have $M, j\models \phi$.

\end{definition}

\begin{definition}[\textbf{semantic consequence}]
Let $\L = (\mathscr{P}, \mathscr{F}, \arity)$ be a signature of a language and let $\Gamma \subset \F(\L)$ and $\phi\in\F(\L)$. We define that $\Gamma \models \phi$ iff for any $\L$-structure $\M$, if $\M\models \psi$ for all $\psi\in\Gamma$, then $\M\models \phi$.
\end{definition}

\begin{definition}[\textbf{tautology}]
Let $\L = (\mathscr{P}, \mathscr{F}, \arity)$ be a signature of a language and let $\phi\in\F(\L)$. We define that $\models \phi$ iff $\emptyset \models \phi$.
\end{definition}

\begin{lemma}
\label{term-evaluation-independence}
Let $\L = (\mathscr{P}, \mathscr{F}, \arity)$ be a signature of a language and $A$ be an arbitrary mathematical domain, let $\M$ be an $\L$-structure with domain $A$ and $j_1, j_2:\V\to A$. Let $\tau\in\T(\L)$.
If $j_1\big(\var(\tau)\big) = j_2\big(\var(\tau)\big)$, then
\begin{equation}
\tau^{\M, j_1} = \tau^{\M, j_2}.
\end{equation}
\end{lemma}

\begin{proof}
We proceed by structural induction on the term $\tau$.

\textbf{Case 1:} $\tau \sxeq v$ where $v\in\V$.

Then $\var(\tau) = \{v\}$. Since $j_1(\var(\tau)) = j_2(\var(\tau))$, we have $j_1(v) = j_2(v)$. By the definition of term evaluation,
\begin{equation*}
\tau^{\M, j_1} = j_1(v) = j_2(v) = \tau^{\M, j_2}.
\end{equation*}

\textbf{Case 2:} $\tau \sxeq f(\bm{\sigma})$ where $\bm{\sigma}\in\T(\L)^*$ and $f\in\mathscr{F}_{|\bm{\sigma}|}$.

Then $\var(\tau) = \var(\bm{\sigma}) = \bigcup_{i=1}^{|\bm{\sigma}|} \var(\sigma_i)$. Since $j_1(\var(\tau)) = j_2(\var(\tau))$, we have $j_1(\var(\sigma_i)) = j_2(\var(\sigma_i))$ for all $i \in \{1, \dots, |\bm{\sigma}|\}$. By inductive hypothesis, $\sigma_i^{\M, j_1} = \sigma_i^{\M, j_2}$ for all $i$. Therefore, $\bm{\sigma}^{\M, j_1} = \bm{\sigma}^{\M, j_2}$. By the definition of term evaluation,
\begin{equation*}
\tau^{\M, j_1} = f^\M(\bm{\sigma}^{\M, j_1}) = f^\M(\bm{\sigma}^{\M, j_2}) = \tau^{\M, j_2}.
\end{equation*}

This completes the proof by structural induction.
\end{proof}

\begin{theorem}
\label{signature-of-formula}
Let $\L = (\mathscr{P}, \mathscr{F}, \arity)$ be a signature of a language and $A$ be an arbitrary mathematical domain, let $\M$ be an $\L$-structure with domain $A$ and $j_1, j_2:\V\to A$. Let $\phi\in\F(\L)$.
If $j_1\big(\free(\phi)\big) = j_2\big(\free(\phi)\big)$, then

\begin{equation}
\phi^{\M, j_1} = \phi^{\M, j_2}.
\end{equation}
\end{theorem}

\begin{proof}
We proceed by structural induction on the formula $\phi$.

\textbf{Case 1:} $\phi \sxeq p(\bm{\tau})$ where $\bm{\tau}\in\T(\L)^*$ and $p\in\mathscr{P}_{|\bm{\tau}|}$.

Then $\free(\phi) = \var(\phi) = \var(\bm{\tau}) = \bigcup_{i=1}^{|\bm{\tau}|} \var(\tau_i)$. Since $j_1(\free(\phi)) = j_2(\free(\phi))$, we have $j_1(\var(\tau_i)) = j_2(\var(\tau_i))$ for all $i$. By Lemma \ref{term-evaluation-independence}, $\tau_i^{\M, j_1} = \tau_i^{\M, j_2}$ for all $i$. Therefore, $\bm{\tau}^{\M, j_1} = \bm{\tau}^{\M, j_2}$. By the definition of formula evaluation,
\begin{equation*}
\phi^{\M, j_1} = p^\M(\bm{\tau}^{\M, j_1}) = p^\M(\bm{\tau}^{\M, j_2}) = \phi^{\M, j_2}.
\end{equation*}

\textbf{Case 2:} $\phi \sxeq \alpha \to \beta$ where $\alpha, \beta\in\B(\L)$.

Then $\free(\phi) = \free(\alpha) \cup \free(\beta)$. Since $j_1(\free(\phi)) = j_2(\free(\phi))$, we have $j_1(\free(\alpha)) = j_2(\free(\alpha))$ and $j_1(\free(\beta)) = j_2(\free(\beta))$. By inductive hypothesis, $\alpha^{\M, j_1} = \alpha^{\M, j_2}$ and $\beta^{\M, j_1} = \beta^{\M, j_2}$. By the definition of formula evaluation,
\begin{equation*}
\phi^{\M, j_1} = 1 - \alpha^{\M, j_1}(1 - \beta^{\M, j_1}) = 1 - \alpha^{\M, j_2}(1 - \beta^{\M, j_2}) = \phi^{\M, j_2}.
\end{equation*}

\textbf{Case 3:} $\phi \sxeq \forall x. \psi$ where $x\in\V$ and $\psi\in\B(\L)$.

Then $\free(\phi) = \free(\psi) \setminus \{x\}$. For any $a\in A$, the assignments $j_1\cfrac{a}{x}$ and $j_2\cfrac{a}{x}$ agree on $\free(\psi)$ because:
\begin{itemize}
\item For $v \in \free(\psi) \setminus \{x\}$, we have $v \in \free(\phi)$, so $(j_1\cfrac{a}{x})(v) = j_1(v) = j_2(v) = (j_2\cfrac{a}{x})(v)$.
\item For $v = x$, we have $(j_1\cfrac{a}{x})(x) = a = (j_2\cfrac{a}{x})(x)$.
\end{itemize}
Thus $j_1\cfrac{a}{x}(\free(\psi)) = j_2\cfrac{a}{x}(\free(\psi))$. By inductive hypothesis, $\psi^{\M, j_1\cfrac{a}{x}} = \psi^{\M, j_2\cfrac{a}{x}}$ for all $a\in A$. By the definition of formula evaluation,
\begin{equation*}
\phi^{\M, j_1} = \prod_{a\in A} \psi^{\M, j_1\cfrac{a}{x}} = \prod_{a\in A} \psi^{\M, j_2\cfrac{a}{x}} = \phi^{\M, j_2}.
\end{equation*}

This completes the proof by structural induction.
\end{proof}

\section{Syntactic Sugar}

In this section, we introduce common logical connectives as syntactic abbreviations of formulas using implication and the false predicate $\false$.

\begin{definition}[\textbf{logical connectives as syntactic sugar}]
Let $\L = (\mathscr{P}, \mathscr{F}, \arity)$ be a signature of a language and let $\alpha, \beta\in\B(\L)$. We define the following syntactic abbreviations:
\begin{align}
\sim\alpha &\defeq \alpha \to \false, \\
\alpha \land \beta &\defeq (\alpha \to (\beta \to \false)) \to \false, \\
\alpha \lor \beta &\defeq (\alpha \to \false) \to \beta, \\
\alpha \liff \beta &\defeq (\alpha \to \beta) \land (\beta \to \alpha).
\end{align}

For any $v\in\V$ and $\psi\in\B(\L)$, we also define:
\begin{equation}
\exists v. \psi \defeq \sim(\forall v. \sim\psi).
\end{equation}
\end{definition}

The following theorems characterize the semantic behavior of these connectives.

\begin{theorem}[\textbf{negation evaluation}]
\label{negation-evaluation}
Let $\L = (\mathscr{P}, \mathscr{F}, \arity)$ be a signature of a language and $A$ be an arbitrary mathematical domain, let $\M$ be an $\L$-structure with domain $A$ and $j:\V\to A$. Let $\alpha\in\B(\L)$, then
\begin{equation}
(\sim\alpha)^{\M, j} = 1 - \alpha^{\M, j}.
\end{equation}
\end{theorem}

\begin{proof}
By definition, $\sim\alpha \sxeq \alpha \to \false$. By the definition of formula evaluation for implication and the fact that $\false^{\M, j} = 0$,
\begin{equation}
(\sim\alpha)^{\M, j} = (\alpha \to \false)^{\M, j} = 1 - \alpha^{\M, j}(1 - \false^{\M, j}) = 1 - \alpha^{\M, j}(1 - 0) = 1 - \alpha^{\M, j}.
\end{equation}
\end{proof}

\begin{theorem}[\textbf{conjunction evaluation}]
\label{conjunction-evaluation}
Let $\L = (\mathscr{P}, \mathscr{F}, \arity)$ be a signature of a language and $A$ be an arbitrary mathematical domain, let $\M$ be an $\L$-structure with domain $A$ and $j:\V\to A$. Let $\alpha, \beta\in\B(\L)$, then
\begin{equation}
(\alpha \land \beta)^{\M, j} = \alpha^{\M, j} \cdot \beta^{\M, j}.
\end{equation}
\end{theorem}

\begin{proof}
By definition, $\alpha \land \beta \sxeq (\alpha \to (\beta \to \false)) \to \false$. By the definition of formula evaluation for implication and the fact that $\false^{\M, j} = 0$,
\begin{align*}
(\alpha \land \beta)^{\M, j}
&= ((\alpha \to (\beta \to \false)) \to \false)^{\M, j} \\
&= 1 - (\alpha \to (\beta \to \false))^{\M, j}(1 - 0) \\
&= 1 - (\alpha \to (\beta \to \false))^{\M, j}.
\end{align*}

Now, expanding the inner implication:
\begin{align*}
(\alpha \to (\beta \to \false))^{\M, j}
&= 1 - \alpha^{\M, j}(1 - (\beta \to \false)^{\M, j}) \\
&= 1 - \alpha^{\M, j}(1 - (1 - \beta^{\M, j}(1 - 0))) \\
&= 1 - \alpha^{\M, j}(1 - (1 - \beta^{\M, j})) \\
&= 1 - \alpha^{\M, j} \cdot \beta^{\M, j}.
\end{align*}

Therefore,
\begin{equation}
(\alpha \land \beta)^{\M, j} = 1 - (1 - \alpha^{\M, j} \cdot \beta^{\M, j}) = \alpha^{\M, j} \cdot \beta^{\M, j}.
\end{equation}
\end{proof}

\begin{theorem}[\textbf{disjunction evaluation}]
\label{disjunction-evaluation}
Let $\L = (\mathscr{P}, \mathscr{F}, \arity)$ be a signature of a language and $A$ be an arbitrary mathematical domain, let $\M$ be an $\L$-structure with domain $A$ and $j:\V\to A$. Let $\alpha, \beta\in\B(\L)$, then
\begin{equation}
(\alpha \lor \beta)^{\M, j} = 1 - (1 - \alpha^{\M, j})(1 - \beta^{\M, j}).
\end{equation}
\end{theorem}

\begin{proof}
By definition, $\alpha \lor \beta \sxeq (\alpha \to \false) \to \beta$. By the definition of formula evaluation for implication:
\begin{align*}
(\alpha \lor \beta)^{\M, j}
&= ((\alpha \to \false) \to \beta)^{\M, j} \\
&= 1 - (\alpha \to \false)^{\M, j}(1 - \beta^{\M, j}) \\
&= 1 - (1 - \alpha^{\M, j})(1 - \beta^{\M, j}).
\end{align*}
\end{proof}

\begin{theorem}[\textbf{biconditional evaluation}]
\label{biconditional-evaluation}
Let $\L = (\mathscr{P}, \mathscr{F}, \arity)$ be a signature of a language and $A$ be an arbitrary mathematical domain, let $\M$ be an $\L$-structure with domain $A$ and $j:\V\to A$. Let $\alpha, \beta\in\B(\L)$, then
\begin{equation}
(\alpha \liff \beta)^{\M, j} = 1 \text{ if and only if } \alpha^{\M, j} = \beta^{\M, j}.
\end{equation}
\end{theorem}

\begin{proof}
By definition, $\alpha \liff \beta \sxeq (\alpha \to \beta) \land (\beta \to \alpha)$. By Theorem \ref{conjunction-evaluation},
\begin{equation}
(\alpha \liff \beta)^{\M, j} = (\alpha \to \beta)^{\M, j} \cdot (\beta \to \alpha)^{\M, j}.
\end{equation}

Expanding using the definition of implication:
\begin{align*}
(\alpha \liff \beta)^{\M, j}
&= (1 - \alpha^{\M, j}(1 - \beta^{\M, j}))(1 - \beta^{\M, j}(1 - \alpha^{\M, j})).
\end{align*}

This product equals 1 if and only if both factors equal 1, which happens if and only if:
\begin{itemize}
\item $\alpha^{\M, j}(1 - \beta^{\M, j}) = 0$, and
\item $\beta^{\M, j}(1 - \alpha^{\M, j}) = 0$.
\end{itemize}

These conditions hold if and only if $\alpha^{\M, j} = \beta^{\M, j}$.
\end{proof}

\begin{theorem}[\textbf{existential quantifier evaluation}]
\label{existential-evaluation}
Let $\L = (\mathscr{P}, \mathscr{F}, \arity)$ be a signature of a language and $A$ be an arbitrary mathematical domain, let $\M$ be an $\L$-structure with domain $A$ and $j:\V\to A$. Let $v\in\V$ and $\psi\in\B(\L)$, then
\begin{equation}
(\exists v. \psi)^{\M, j} = 1 - \prod_{a\in A} (1 - \psi^{\M, j\cfrac{a}{v}}).
\end{equation}
\end{theorem}

\begin{proof}
By definition, $\exists v. \psi \sxeq \sim(\forall v. \sim\psi)$. By Theorem \ref{negation-evaluation},
\begin{equation}
(\exists v. \psi)^{\M, j} = 1 - (\forall v. \sim\psi)^{\M, j}.
\end{equation}

By the definition of formula evaluation for universal quantification,
\begin{equation}
(\forall v. \sim\psi)^{\M, j} = \prod_{a\in A} (\sim\psi)^{\M, j\cfrac{a}{v}}.
\end{equation}

By Theorem \ref{negation-evaluation}, for each $a\in A$,
\begin{equation}
(\sim\psi)^{\M, j\cfrac{a}{v}} = 1 - \psi^{\M, j\cfrac{a}{v}}.
\end{equation}

Therefore,
\begin{equation}
(\exists v. \psi)^{\M, j} = 1 - \prod_{a\in A} (1 - \psi^{\M, j\cfrac{a}{v}}).
\end{equation}
\end{proof}

\section{Substitutions}
\subsection{Syntax}



\begin{definition}[\textbf{recursive: substitution in a term}]
Let $\L = (\mathscr{P}, \mathscr{F}, \arity)$ be a signature of a language.
Let $k$ be a positive integer and
let $\bm{x}\in \dis{\V}{k}$ and $\bm{\tau}\in\T(\L)^k$.\\  
Let $\sigma\in\T(\L)$. 
We will define recursively $\sigma(\bm{x}/\bm{\tau})$.
Let's establish that $\sigma(\emptyset/\emptyset) \defeq \sigma$.
\begin{enumerate}
\item If $\sigma \sxeq x_i$ for some $i\in\{1, \dots, k\}$, then $\sigma(\bm{x}/\bm{\tau}) \defeq \tau_i$.
\item If $\sigma \sxeq v$ where $v\in\V$ and $v\not=x$, then $\sigma(\bm{x}/\bm{\tau}) \defeq v$.
\item If $\tau \sxeq f(\bm{\eta})$ where $\bm{\eta}\in\T(\L)^*$ and $f\in \mathscr{F}_{|\bm{\eta}|}$, then $\sigma(\bm{x}/\bm{\tau}) \defeq f(\bm{\eta}(\bm{x}/\bm{\tau}))$
where
\begin{equation}
\bm{\eta}(\bm{x}/\bm{\tau}) \defeq
\begin{cases}
(\eta_1(\bm{x}/\bm{\tau}), \dots, \eta_k(\bm{x}/\bm{\tau})) \text{ for } k = |\bm{\eta}| > 0,\\
\emptyset \text{ for } |\bm{\eta}| = 0.
\end{cases}
\end{equation}
\end{enumerate}

\end{definition}

\begin{definition}[\textbf{recursive: admissible substitution in a formula}]
Let $\L = (\mathscr{P}, \mathscr{F}, \arity)$ be a signature of a language.
Let $k$ be a positive integer and
let $\bm{x}\in \dis{\V}{k}$ and $\bm{\tau}\in\T(\L)^k$.\\

Let $\phi\in\F(\L)$. 
We will define recursively $\sigma(\bm{x}/\bm{\tau})$.
Let's establish that $\phi(\emptyset/\emptyset) \defeq \phi$ and is \textbf{admissible}.
\begin{enumerate}
\item If $\phi \sxeq p(\bm{\eta})$ where $\bm{\eta}\in\T(\L)^*$ and $p\in \mathscr{P}_{|\bm{\eta}|}$, then $\phi(\bm{x}/\bm{\tau}) \defeq p(\bm{\eta}(\bm{x}/\bm{\tau}))$ and is \textbf{admissible}.
\item If $\phi \sxeq \alpha \to \beta$ where $\alpha, \beta\in\B(\L)$ and $\alpha(\bm{x}/\bm{\tau}), \beta(\bm{x}/\bm{\tau})$ are \textbf{admissible}, then $\phi(\bm{x}/\bm{\tau}) \defeq \alpha(\bm{x}/\bm{\tau}) \to \beta(\bm{x}/\bm{\tau})$ and is \textbf{admissible}.
\item If $\phi \sxeq \forall u. \psi$ where $u\in \V$ and $\psi\in\B(\L)$:\\
Let $\bm{v}$ be $\bm{x}$ with $x_i$ removed when $x_i=u$ or $x_i\not\in\free(\psi)$. And let $\bm{\sigma}$ be $\bm{\tau}$ with $\tau_i$ removed for which $x_i$ was removed from $\bm{x}$.

\begin{enumerate}
\item If $\psi(\bm{v}/\bm{\sigma})$ is \textbf{admissible} and 
$u\not\in\var(\bm{\sigma})$, then $\phi(\bm{x}/\bm{\tau}) \defeq \forall u. \psi(\bm{v}/\bm{\sigma})$ and is \textbf{admissable}.
\item Otherwise $\phi(\bm{x}/\bm{\tau})$ is \textbf{not admissible}.
\end{enumerate} 
\end{enumerate}
\end{definition}

\begin{definition}[\textbf{recursive: admissible predicate substitution}]
\label{predicate-substitution-definition}
Let $\L = (\mathscr{P}, \mathscr{F}, \arity)$ be a signature of a language.
Let $k$ be a non-negative integer and
let $\bm{x}\in \dis{\V}{k}$.
Let $q\in \mathscr{P}_k$ and $\phi \in\F(\L)$.\\

Let $\psi\in\F(\L)$.
We will define recursively $\psi(q(\bm{x})/\phi)$.

\begin{enumerate}
\item If $\psi \sxeq p(\bm{\eta})$ where $\bm{\eta}\in\T(\L)^*$ and $p\in \mathscr{P}_{|\bm{\eta}|}$.
\begin{enumerate}
\item If $p = q$ and $\phi(\bm{x}/\bm{\eta})$ is admissible, then $\psi(q(\bm{x})/\phi) \defeq \phi(\bm{x}/\bm{\eta})$ and is \textbf{admissible}.
\item If $p \not= q$, then $\psi(q(\bm{x})/\phi) \defeq \psi$ and is \textbf{admissible}.
\end{enumerate}
\item If $\psi \sxeq \alpha \to \beta$ where $\alpha, \beta\in\B(\L)$
and $\alpha(q(\bm{x})/\phi)$, $\beta(q(\bm{x})/\phi)$ are \textbf{admissible}, then
$\psi(q(\bm{x})/\phi) \defeq \alpha(q(\bm{x})/\phi)\to\beta(q(\bm{x})/\phi)$ and is \textbf{admissible}.
\item If $\psi \sxeq \forall u. \gamma$ where $u\in \V$ and $\gamma\in\B(\L)$:
\begin{enumerate}
\item If $q$ occurs in $\gamma$, $u\not\in \free(\phi)\setminus \bm{x}$, $\gamma(q(\bm{x})/\phi)$ is \textbf{admissible}, then 
$\psi(q(\bm{x})/\phi) \defeq \forall u. \gamma(q(\bm{x})/\phi)$ and is \textbf{admissible}.
\item If $q$ does not occur in $\gamma$, then $\psi(q(\bm{x})/\phi) \defeq \forall u. \gamma$ and is \textbf{admissible}.
\item Otherwise $\psi(q(\bm{x})/\phi)$ is \textbf{not admissible}.
\end{enumerate}
\end{enumerate}
\end{definition}


\subsection{Semantics}

\begin{lemma}
\label{term-substitution-lemma}
Let  $\L = (\mathscr{P}, \mathscr{F}, \arity)$ be a signature of a language and $A$ be an arbitrary mathematical domain. Let $\M$ be an $\L$-structure with domain $A$ and $j:\V\to A$ be an assignment. 

Let $k$ be a positive integer and
let $\bm{x}\in \dis{\V}{k}$ and $\bm{\tau}\in\T(\L)^k$.\\


Then for any term $\sigma$, we have
\begin{equation}
\sigma^{\M, j\cfrac{\bm{\tau}^{\M, j}}{\bm{x}}} = \sigma(\bm{x}/\bm{\tau})^{\M, j}.
\end{equation}
\end{lemma}

\begin{proof}
We proceed by structural induction on the term $\sigma$.

\textbf{Case 1:} $\sigma \sxeq x_i$ for some $i\in\{1, \dots, k\}$.

Then $\sigma(\bm{x}/\bm{\tau}) \sxeq \tau_i$ and
\begin{equation*}
\sigma^{\M, j\cfrac{\bm{\tau}^{\M, j}}{\bm{x}}} = (j\cfrac{\bm{\tau}^{\M, j}}{\bm{x}})(x_i) = \tau_i^{\M, j} = \sigma(\bm{x}/\bm{\tau})^{\M, j}.
\end{equation*}

\textbf{Case 2:} $\sigma \sxeq v$ where $v\in\V$ and $v\neq x$.

Then $\sigma(x/\tau) \sxeq v$ and
\begin{equation*}
\sigma^{\M, j\cfrac{\bm{\tau}^{\M, j}}{\bm{x}}} = (j\cfrac{\bm{\tau}^{\M, j}}{\bm{x}})(v) = j(v) = v^{\M, j} = (\bm{x}/\bm{\tau})^{\M, j}.
\end{equation*}

\textbf{Case 3:} $\sigma \sxeq f(\bm{\eta})$ where $\bm{\eta}\in\T(\L)^*$ and $f\in \mathscr{F}_{|\bm{\eta}|}$.

Then $\sigma(\bm{x}/\bm{\tau}) \sxeq f(\bm{\eta}(\bm{x}/\bm{\tau}))$. By inductive hypothesis, $\bm{\eta}^{\M, j\cfrac{\bm{\tau}^{(\M, j)}}{\bm{x}}} = \bm{\eta}(\bm{x}/\bm{\tau})^{\M, j}$. Thus
\begin{align*}
\sigma^{\M, j\cfrac{\bm{\tau}^{\M, j}}{\bm{x}}} &= f^\M\bigg(\bm{\eta}^{\M, j\cfrac{\bm{\tau}^{\M, j}}{\bm{x}}}\bigg) = f^\M\bigg(\bm{\eta}(\bm{x}/\bm{\tau})^{\M, j}\bigg) \\
&= f(\bm{\eta}(\bm{x}/\bm{\tau}))^{\M, j} = \sigma(\bm{x}/\bm{\tau})^{\M, j}.\\
\end{align*}

This completes the proof by structural induction.
\end{proof}

\begin{theorem}
\label{admissible-theorem}
Let  $\L = (\mathscr{P}, \mathscr{F}, \arity)$ be a signature of a language. Let $\phi\in\F(\L)$.
Let $k$ be a positive integer and
let $\bm{x}\in \dis{\V}{k}$ and $\bm{\tau}\in\T(\L)^k$.
If $\phi(\bm{x}/\bm{\tau})$ is admissible, then for any $\L$-structure $\M$ with domain $A$ and for any assignment of variables $j:\V\to A$ we have
\begin{equation}
(\M, j\cfrac{\bm{\tau}^{\M, j}}{\bm{x}})\models \phi \iff \M, j\models \phi(\bm{x}/\bm{\tau}).
\end{equation}
\end{theorem}

\begin{proof}
We proceed by structural induction on $\phi\in\F(\L)$.

\textbf{Case 1:} $\phi \sxeq p(\bm{\eta})$ where $\bm{\eta}\in\T(\L)^*$ and $p\in \mathscr{P}_{|\bm{\eta}|}$.

By the definition of substitution in well formed formula, $\phi(\bm{x}/\bm{\tau}) \sxeq p(\bm{\eta(\bm{x}/\bm{\tau}}))$. By Lemma \ref{term-substitution-lemma}, we have
$\bm{\eta}^{\M, j\cfrac{\bm{\tau}^{\M, j}}{\bm{x}}} = \bm{\eta}(\bm{x}/\bm{\tau})^{\M, j}$. Therefore,
\begin{align*}
\phi^{\M, j\cfrac{\tau^{\M, j}}{x}} = p^\M\bigg(\bm{\eta}^{\M, j\cfrac{\bm{\tau}^{\M, j}}{\bm{x}}}\bigg) = p^\M(\bm{\eta}(\bm{x}/\bm{\tau})^{\M, j}) 
= p(\bm{\eta}(\bm{x}/\bm{\tau}))^{\M, j} 
= \phi(\bm{x}/\bm{\tau})^{\M, j}.
\end{align*}

\textbf{Case 2:} $\phi \sxeq \alpha \to \beta$ where $\alpha, \beta\in\B(\L)$.

By the definition of substitution in well formed formula, $\phi(\bm{x}/\bm{\tau}) \sxeq \alpha(\bm{x}/\bm{\tau}) \to \beta(\bm{x}/\bm{\tau})$. By inductive hypothesis,
\begin{align*}
(\M, j\cfrac{\bm{\tau}^{\M, j}}{\bm{x}})\models \alpha &\iff \M, j\models \alpha(\bm{x}/\bm{\tau}), \\
(\M, j\cfrac{\bm{\tau}^{\M, j}}{\bm{x}})\models \beta &\iff \M, j\models \beta(\bm{x}/\bm{\tau}).
\end{align*}

By the definition of formula evaluation,
\begin{align*}
\phi^{\M, j\cfrac{\bm{\tau}^{\M, j}}{\bm{x}}} 
&= 1 - \alpha^{\M, j\cfrac{\bm{\tau}^{\M, j}}{\bm{x}}}\bigg(1 - \beta^{\M, j\cfrac{\bm{\tau}^{\M, j}}{\bm{x}}}\bigg) 
= 1 - (\alpha(\bm{x}/\bm{\tau}))^{\M, j}(1 - (\beta(\bm{x}/\bm{\tau}))^{\M, j}) = 1 \\
&= (\alpha(\bm{x}/\bm{\tau}) \to \beta(\bm{x}/\bm{\tau}))^{\M, j} = (\phi(\bm{x}/\bm{\tau}))^{\M, j}\\
\end{align*}

\textbf{Case 3:} $\phi \sxeq \forall u. \psi$ where $u\in \V$ and $\psi\in\B(\L)$:\\
Let $\bm{v}$ be $\bm{x}$ with $x_i$ removed when $x_i=u$ or $x_i\not\in\free(\psi)$. And let $\bm{\sigma}$ be $\bm{\tau}$ with $\tau_i$ removed for which $x_i$ was removed from $\bm{x}$.

Since $\phi(\bm{x}/\bm{\tau})$ is admissible, $\psi(\bm{v}/\bm{\sigma})$ is \textbf{admissible} and 
$u\not\in\var(\bm{\sigma})$, and we have $\phi(\bm{x}/\bm{\tau}) \sxeq \forall u. \psi(\bm{v}/\bm{\sigma})$.
Thus
\begin{equation}
\phi^{\M, j\cfrac{\bm{\tau}^{\M, j}}{\bm{x}}} 
= \prod_{a\in A} \psi^{\M, j\cfrac{\bm{\tau}^{\M, j}}{\bm{x}}\cfrac{a}{u}}
\end{equation}

By choice of $\bm{v}$, $j\cfrac{\bm{\tau}^{\M, j}}{\bm{x}}\cfrac{a}{u}$
and 
$j\cfrac{\bm{\sigma}^{\M, j}}{\bm{v}}\cfrac{a}{u}$ agrees on $\free(\psi)$ 
therefore by Theorem \ref{signature-of-formula} we have  

\begin{align*}
\prod_{a\in A} \psi^{\M, j\cfrac{\bm{\tau}^{\M, j}}{\bm{x}}\cfrac{a}{u}}
&= \prod_{a\in A} \psi^{\M, j\cfrac{\bm{\sigma}^{\M, j}}{\bm{v}}\cfrac{a}{u}}\\
&= \prod_{a\in A} \psi^{\M, j\cfrac{a}{u}\cfrac{\bm{\sigma}^{\M, j}}{\bm{v}}}
\quad 
\big(
\text{since $u$ not in $\bm{v}$}
\big)
\\
&= \prod_{a\in A} \psi(\bm{v}/\bm{\sigma})^{\M, j\cfrac{a}{u}}
\quad
\big(
\text{by inductive hypothesis}
\big)
\\
&= (\forall u. \psi(\bm{v}/\bm{\sigma}))^{\M, j} = \phi(\bm{x}/\bm{\tau})^{M,j}.
\end{align*}

This completes the proof by structural induction.
\end{proof}

\begin{lemma}
\label{predicate-substitution-lemma}
Let $\L=(\mathscr{P}, \mathscr{F}, \arity)$ be a signature of a language and let $q\in \mathscr{P}$. Let $\phi\in\F(\L)$ and $\bm{x}\in \dis{\V}{k}$. Let $j_0: \free(\phi)\setminus \bm{x} \to A$, where $A$ is a domain of an $\L$-structure $M$ for which
\begin{equation}
\label{phi-q-equivalence}
M, j \models q(\bm{x}) \text{ iff } M, j \models \phi
\end{equation}
for any $j$ that is an extention of $j_0$. If $\psi\in\F(\L)$ and $\psi(q(\bm{x})/\phi)$ is admissible, then
\begin{equation}
M, j \models \psi \text{ iff } M, j \models \psi(q(\bm{x})/\phi)
\end{equation}
for any $j$ that is an extention of $j_0$.
\end{lemma}

\begin{proof}
We proceed by structural induction on $\psi\in\F(\L)$ following Definition~\ref{predicate-substitution-definition}.

Take any $j: \V \to A$ which is an extension of $j_0$.

\textbf{Case 1:} $\psi \sxeq p(\bm{\eta})$ where $\bm{\eta}\in\T(\L)^*$ and $p\in \mathscr{P}_{|\bm{\eta}|}$.

\textbf{Subcase 1a:} $p = q$ and $\phi(\bm{x}/\bm{\eta})$ is admissible.

Then by Definition~\ref{predicate-substitution-definition}, $\psi(q(\bm{x})/\phi) \sxeq \phi(\bm{x}/\bm{\eta})$.
By the definition of formula evaluation,
\begin{equation*}
\psi^{M, j} = q^M(\bm{\eta}^{M, j}) = q(\bm{x})^{M, j\cfrac{\bm{\eta}^{M, j}}{\bm{x}}}.
\end{equation*}

Since $j\cfrac{\bm{\eta}^{M, j}}{\bm{x}}$ is an extension of $j_0$, by (\ref{phi-q-equivalence}), we have

\begin{equation*}
q(\bm{x})^{M, j\cfrac{\bm{\eta}^{M, j}}{\bm{x}}} = \phi^{M, j\cfrac{\bm{\eta}^{M, j}}{\bm{x}}}.
\end{equation*}
By Theorem~\ref{admissible-theorem}, since $\phi(\bm{x}/\bm{\eta})$ is admissible,
\begin{equation*}
\phi^{M, j\cfrac{\bm{\eta}^{M, j}}{\bm{x}}} = \phi(\bm{x}/\bm{\eta})^{M, j}.
\end{equation*}
Therefore, $\psi^{M, j} = \phi(\bm{x}/\bm{\eta})^{M, j} = \psi(q(\bm{x})/\phi)^{M, j}$.

\textbf{Subcase 1b:} $p \neq q$.

Then by Definition~\ref{predicate-substitution-definition}, $\psi(q(\bm{x})/\phi) \sxeq \psi$, so $\psi^{M, j} = \psi(q(\bm{x})/\phi)^{M, j}$ holds trivially.

\textbf{Case 2:} $\psi \sxeq \alpha \to \beta$ where $\alpha, \beta\in\B(\L)$ and $\alpha(q(\bm{x})/\phi)$, $\beta(q(\bm{x})/\phi)$ are admissible.

Then by Definition~\ref{predicate-substitution-definition}, $\psi(q(\bm{x})/\phi) \sxeq \alpha(q(\bm{x})/\phi)\to\beta(q(\bm{x})/\phi)$.
By inductive hypothesis, $\alpha^{M, j} = \alpha(q(\bm{x})/\phi)^{M, j}$ and $\beta^{M, j} = \beta(q(\bm{x})/\phi)^{M, j}$.
By the definition of formula evaluation,
\begin{align*}
\psi^{M, j} &= 1 - \alpha^{M, j}(1 - \beta^{M, j}) \\
&= 1 - \alpha(q(\bm{x})/\phi)^{M, j}(1 - \beta(q(\bm{x})/\phi)^{M, j}) \\
&= (\alpha(q(\bm{x})/\phi) \to \beta(q(\bm{x})/\phi))^{M, j} \\
&= \psi(q(\bm{x})/\phi)^{M, j}.
\end{align*}

\textbf{Case 3:} $\psi \sxeq \forall u. \gamma$ where $u\in \V$ and $\gamma\in\B(\L)$.

\textbf{Subcase 3a:} $q$ does not occur in $\gamma$.

Then by Definition~\ref{predicate-substitution-definition}, $\psi(q(\bm{x})/\phi) \sxeq \forall u. \gamma = \psi$, so $\psi^{M, j} = \psi(q(\bm{x})/\phi)^{M, j}$ holds trivially.

\textbf{Subcase 3b:} $q$ occurs in $\gamma$. Since $\psi(q(\bm{x})/\phi)$ is admissible, $u\not\in \free(\phi)\setminus \bm{x}$ and $\gamma(q(\bm{x})/\phi)$ is admissible.

Then by Definition~\ref{predicate-substitution-definition}, $\psi(q(\bm{x})/\phi) \sxeq \forall u. \gamma(q(\bm{x})/\phi)$.

For any $a\in A$, let $j' = j\cfrac{a}{u}$.
Since $u\not\in \free(\phi)\setminus \bm{x}$, the assignment $j'$ is also an extension of $j_0$.
By inductive hypothesis, $\gamma^{M, j'} = \gamma(q(\bm{x})/\phi)^{M, j'}$.

This holds for all $a\in A$, therefore by the definition of formula evaluation,
\begin{align*}
\psi^{M, j} &= \prod_{a\in A} \gamma^{M, j\cfrac{a}{u}} \\
&= \prod_{a\in A} \gamma(q(\bm{x})/\phi)^{M, j\cfrac{a}{u}} \\
&= (\forall u. \gamma(q(\bm{x})/\phi))^{M, j} \\
&= \psi(q(\bm{x})/\phi)^{M, j}.
\end{align*}

This completes the proof by structural induction.
\end{proof}

\begin{theorem}
\label{predicate-substitution-theorem}
Let $\L=(\mathscr{P}, \mathscr{F}, \arity)$ be a signature of a language and let $q\in \mathscr{P}$. Let $\phi\in\F(\L)$ and $\bm{x}\in \dis{\V}{k}$. 
If $\models \psi$ and $\psi(q(\bm{x})/\phi)$ is admissible, then
$\models \psi(q(\bm{x})/\phi)$.
\end{theorem}
\begin{proof}
Let's introduce a new predicate symbol $q'$ such that $q'\not\in \mathscr{P}$.
Let $\L'=(\mathscr{P}\cup\{q'\}, \mathscr{F}, \arity')$ such that $\arity'$ is an extension of $\arity$ such that $\arity'(q') = \arity(q)$. Let $k = \arity(q)$. Let $\psi'$ be $\psi$ with each $q$ symbol replaced by $q'$. Since $\models \psi$, also $\models \psi'$.
Take any non-empty domain $A$.
Take any $\L$-structure $\M$ with domain $A$.
Take any $j_0:\free(\phi)\setminus \bm{x}\to A$.
We will define an $\L'$ structure.
\begin{equation}
\M'_{j_0}(s) =
\begin{cases}
\M'_{}(s) \text{ for } s\in \mathscr{P}\cup \mathscr{F},\\
\big(A^k\ni\bm{a}\mapsto\phi^{M, l\cfrac{\bm{a}}{\bm{x}}}\big) \text{ for } s = q'.
\end{cases}
\end{equation}
Where $l: \V\to A$ is some fixed extension of $j_0$ (note that choice of $l$ does not matter for the above definition).
Take any $j:\V\to A$ which is an extension of $j_0$. Since $\models\psi'$, we have 
$M'_{j_0}, j\models\psi'$.
Note that
\begin{equation}
q'(\bm{x})^{M'_{j_0}, j} = (q')^{M'_{j_0}}(\bm{x}^{M'_{j_0}, j})
= \phi^{M'_{j_0}, l\cfrac{\bm{x}^{M'_{j_0}, j}}{\bm{x}}}.
\end{equation}

Since $l\cfrac{\bm{x}^{M'_{j_0}, j}}{\bm{x}}$ 
and $j\cfrac{\bm{x}^{M'_{j_0}, j}}{\bm{x}}$ 
agrees on $\free(\phi)$, by Theorem \ref{signature-of-formula} we have

\begin{equation}
\phi^{M'_{j_0}, l\cfrac{\bm{x}^{M'_{j_0}, j}}{\bm{x}}} = \phi^{M'_{j_0}, j\cfrac{\bm{x}^{M'_{j_0}, j}}{\bm{x}}} = \phi^{M'_{j_0}, j}.
\end{equation}

We have shown that 
\begin{equation}
q'(\bm{x})^{M'_{j_0}, j} = \phi^{M'_{j_0}, j}.
\end{equation}

Thus by Theorem \ref{predicate-substitution-theorem}, we have 
$M'_{j_0}, j\models \psi'(q(\bm{x})/\phi)$. Note that $\psi'(q'(\bm{x})/\phi) \sxeq \psi(q(\bm{x})/\phi)$ and therefore $\psi'(q'(\bm{x})/\phi)\in\F(\L)$. Then by construction of $M'_{j_0}$, we have $M, j\models \psi'(q'(\bm{x})/\phi)$, and consequently
\begin{equation}
\label{predicate-substitution-for-j}
M, j\models \psi(q(\bm{x})/\phi).
\end{equation}  
Since $j_0:\free(\phi)\setminus \bm{x}\to A$ was arbitrarly chosen, (\ref{predicate-substitution-for-j}) holds for arbirary $j:\V\to A$. This completes the proof.
\end{proof}

\section{Rules Preliminaries}

\begin{theorem}[\textbf{modus ponens}]
\label{modus-ponens}
Let  $\L = (\mathscr{P}, \mathscr{F}, \arity)$ be a signature of a language and $A$ be an arbitrary mathematical domain. Let $\M$ be an $\L$-structure with domain $A$ and $j:\V\to A$ be an assignment Let $\alpha, \beta\in\F(\L)$. If $\M, j\models \alpha$ and $\M, j\models \alpha \to \beta$, then $\M,j\models \beta$.
\end{theorem}

\begin{proof}
We have $\M, j\models \alpha$, which means $\alpha^{\M, j} = 1$.

Since $\Gamma\models \alpha \to \beta$, we have $\M, j\models \alpha \to \beta$, which means $(\alpha \to \beta)^{\M, j} = 1$.

By the definition of formula evaluation for implication,
\begin{equation}
(\alpha \to \beta)^{\M, j} = 1 - \alpha^{\M, j}(1 - \beta^{\M, j}) = 1.
\end{equation}

Since $\alpha^{\M, j} = 1$, we have
\begin{equation}
1 - 1 \cdot (1 - \beta^{\M, j}) = 1,
\end{equation}
which simplifies to $1 - (1 - \beta^{\M, j}) = 1$, hence $\beta^{\M, j} = 1$.

Therefore, $\M, j\models \beta$.
\end{proof}

\begin{theorem}[\textbf{generalization}]
\label{generalization}
Let $\L = (\mathscr{P}, \mathscr{F}, \arity)$ be a signature of a language and let $\M$ be an $\L$-structure. Let $v\in\V$ and $\phi\in\B(\L)$. If $\M\models \phi$, then $\M\models \forall v. \phi$.
\end{theorem}

\begin{proof}
Take any non-empty domain $A$ and any $\L$-structure $\M$ with domain $A$, such that $\M\models \phi$. Take any $j:\V\to A$. We need to show that $\M, j\models \forall v. \phi$.

By the definition of formula evaluation for universal quantification,
\begin{equation}
(\forall v. \phi)^{\M, j} = \prod_{a\in A} \phi^{\M, j\cfrac{a}{v}}.
\end{equation}

Since $\M\models \phi$, we have $\M, j\cfrac{a}{v}\models \phi$, which means $\phi^{\M, j\cfrac{a}{v}} = 1$.

This holds for all $a\in A$, therefore
\begin{equation}
(\forall v. \phi)^{\M, j} = \prod_{a\in A} \phi^{\M, j\cfrac{a}{v}} = \prod_{a\in A} 1 = 1.
\end{equation}

Hence $\M, j\models \forall v. \phi$. And since $j:\V\to A$ was arbitrary, then $\M\models\forall v. \phi$.
\end{proof}

\begin{theorem}
\label{modus-ponens-from-def}
Let $\L=(\mathscr{P}, \mathscr{F}, \arity)$ be a signature of a language and let $q\in \mathscr{P}$. Let $\phi, \psi\in\F(\L)$ and $\Gamma\subset\F(\L)$ such that there is no occurence of $q$ in set of formulas $\Gamma \cup \{\phi, \psi\}$. Let $\bm{x}\in \dis{\V}{k}$. Let $A$ be an non-empty mathematical domain and $j:\V\to A$. If for any $\L$-structre $\M$ with domain $A$ such that $\M, j\models \Gamma$, we have 
$\M, j\models (q(\bm{x}) \liff \phi) \to \psi$, then for any $\L$-structre $\M$ with domain $A$ such that $\M, j\models \Gamma$, we have $\M, j\models \psi$.
\end{theorem}

\begin{proof}
Assume that for any $\L$-structre $\M$ with domain $A$ such that $\M, j\models \Gamma$, we have
\begin{equation}
\label{modus-ponens-from-def-assumption}
\M, j\models (q(\bm{x}) \liff \phi) \to \psi.
\end{equation}

Take an arbitrary $\L$-structre $\M$ with domain $A$ such that $\M, j\models \Gamma$. We will modify the structure $\M$ to create a new structure $\M'$ that differs only in the interpretation of $q$. Namely, 
\begin{equation}
(q)^{\M'}(\bm{a}) = \phi^{\M, j\cfrac{\bm{a}}{\bm{x}}}.
\end{equation}

For all other predicates and functions, $\M'$ agrees with $\M$.

With the structure $\M'$, we have
\begin{equation}
q(\bm{x})^{\M', j} = (q)^{\M'}(\bm{x}^{\M', j}) = \phi^{\M, j\cfrac{\bm{x}^{\M', j}}{\bm{x}}}.
\end{equation}
Since by Definition \ref{term-eval-def} (term evaluation), we have $\bm{x}^{\M', j} = \bm{x}^{\M, j}$, thereforre

\begin{equation}
\phi^{\M, j\cfrac{\bm{x}^{\M', j}}{\bm{x}}} = \phi^{\M, j\cfrac{\bm{x}^{\M, j}}{\bm{x}}}
= \phi^{M, j}.
\end{equation}

Hence $q(\bm{x})^{\M', j} = \phi^{\M, j}$. Since $q$ does not occur in $\phi$, 
we have $\phi^{\M, j} = \phi^{\M', j}$ and consequently $q(\bm{x})^{\M', j} = \phi^{\M', j}$. By Theorem \ref{biconditional-evaluation} this is the same as $(q(\bm{x}) \liff \phi)^{\M', j} = 1$.


Since there is no occurence of $q$ in any formula from $\Gamma$, we have $\M', j\models \Gamma$ and thus by assumption (\ref{modus-ponens-from-def-assumption}), we have $\M', j\models (q(\bm{x}) \liff \phi) \to \psi$, which is the same as
\begin{equation}
((q(\bm{x}) \liff \phi) \to \psi)^{\M', j} = 1.
\end{equation}

This means
\begin{equation}
1 - (q(\bm{x}) \liff \phi)^{\M', j}(1 - \psi^{\M', j}) = 1.
\end{equation}

But since $(q(\bm{x}) \liff \phi)^{\M', j} = 1$, we have
\begin{equation}
1 - 1 \cdot (1 - \psi^{\M', j}) = 1,
\end{equation}
which implies $\psi^{\M', j} = 1$. And since $q$ does not occur in $\psi$, $\psi^{\M, j} = 1$, so $\M, j\models \psi$.
\end{proof}

\end{document}